% !TEX encoding = UTF-8 Unicode
% !TEX TS-program = XeLaTeX
\documentclass [12pt]{article}
\usepackage[Lenny]{fncychap}
\usepackage{mathrsfs}
\usepackage{amsmath}
\usepackage{amssymb}
\usepackage{graphics}
\textwidth=6.5in \textheight=9in \topmargin=1pt \oddsidemargin=0pt
\evensidemargin=0pt
\renewcommand{\baselinestretch}{1.5}
\usepackage{CJK}
\usepackage{fontspec,xltxtra,xunicode}  
\defaultfontfeatures{Mapping=tex-text}  
%\setromanfont{SimSun} %设置中文字体  
\setromanfont{Kai} %设置中文字体  
\XeTeXlinebreaklocale “zh”  
\XeTeXlinebreakskip = 0pt plus 1pt minus 0.1pt %文章内中文自动换行  
\usepackage{multirow}


%-----------------------------------------------




%---------------------------------------------------
\def\hilite<#1>{\temporal<#1>{\color{blue!35}}{\color{magenta}}{\color{blue!75}}}
%% 自定义命令, 源自 beamer_guide. item 逐步显示时, 使已经出现的item、正在显示的item、将要出现的item 呈现不同颜色.
%%%%%%%%%%%%%%%%%%%%%%%%%%%%%%%%%%%%%%%%%%%%%%%%%%%%%%%%%%%%%%%%%%%%%%%%%%%%%%%%%%%%%%%%%%%%%%%%%%%%%%%%
\def\la{\lambda}
\def\bfta{{\bm{\theta}}}
\def\Ta{{\Theta}}
\def\ta{{\theta}}
\def\hta{{\hat{\theta}}}
\def\y1n{{Y_1,\cdots, Y_n}}
\def\x1n{{X_1,\cdots, X_n}}
\def\f12{{\frac{1}{2}}}
\def\barx{{\bar{X}}}
\def\f12{{\frac{1}{2}}}
\def\Sig{{\Sigma}}
\def\sig{{\sigma}}
\def\R{{\mathcal{R}}}
\def\p{{\mathcal{P}}}
\def\f{{\mathcal{F}}}
\def\e{{\epsilon}}
\def\barx{{\bar{X}}}
\def\bx{{\bm{x}}}
\def\bX{{\bm{\mathcal{X}}}}

\def\v#1{\stackrel{#1}{\longrightarrow}}
%---------------------------------------------
\newcommand{\cov}{\mbox{cov}}
\newcommand{\Var}{\mbox{Var}}
\newcommand{\E}{\mbox{E}}
%----------------------------------------------------
\newcommand{\limn}{{\lim\limits_{n\to\infty}}}
\newcommand{\liminfn}{{\liminf\limits_{n\to\infty}}}
\newcommand{\limsupn}{{\limsup\limits_{n\to\infty}}}

\newcommand{\sumx}[1]{\sum\limits_{i=1}^n {#1}_i}
\newcommand{\sumn}[1]{\sum\limits_{#1=1}^n}
%----------------------------------------------------
\newcommand{\timx}[1]{\prod\limits_{i=1}^n {#1}_i}
\newcommand{\timn}[1]{\prod\limits_{#1=1}^n}
%-------------------------------------------------------
\def\dps{\displaystyle}
\def\Om{{\Omega}}
\def\om{{\omega}}
\def\pp{{\mathcal{P}}}
\def\f{{\mathcal{F}}}
\def\R{{\mathcal{R}}}
\def\B{{\mathcal{B}}}
\def\A{{\mathcal{A}}}
\def\X{{\mathcal{X}}}
\def\C{{\mathcal{C}}}
\def\D{{\mathcal{D}}}
\def\O{{\mathcal{O}}}
\def\J{{\mathcal{J}}}
\def\Mu{{\mathcal{U}}}
%------------------------------------------
\def\b{{\beta}}
\def\a{{\alpha}}
\def\e{{\epsilon}}
\def\la{\lambda}
\def\La{\Lambda}
\def\ta{{\theta}}
\def\Ta{{\Theta}}
%-------------------------------------

\def\Sig{\Sigma}
\def\sig{\sigma}

\def\dfrac#1#2{{\displaystyle{#1\over#2}}}
\def\indA#1#2{{A^*_{#1 #2}(t)=a_m}}
\def\sumL {{\sum\limits_{m=1}^L}}
\def\dbN#1{{d \bar{N}(#1)}}


\def\Y#1#2{{Y_{#1#2}(t)}}
\def\Iam#1#2{{I(A^*_{#1#2}=a_m)}}


%------------------------------------------
\newcommand{\xn}[1]{ #1_1, \cdots, #1_n}
\newcommand{\capn}[1]{\cap\limits_{#1=1}^n}
\newcommand{\enorm}[2]{(E({#1}^#2))^{1/#2}}

%\newcommand{\sumx}[1]{\sum\limits_{i=1}^n {#1}_i}
\newcommand{\seqn}[1]{#1_1,#1_2,\cdots}

%%%%%%%%%%%%%%%%%%%%%%%%%%%%
\newtheorem{exam}{example}[section]
\newtheorem{df}{definition}[section]
\newtheorem{thm}{Theorem}[section]
\newtheorem{lem}{Lemma}[section]
\newtheorem{cor}{Corrolary}[section]
\newtheorem{rem}{Remark}[section]
%------------------------------------------------------------------
\begin{document}
 \begin{CJK*}{GBK}{kai}
 %%----------------------- Theorems ---------------------------------------------------------------------
\newtheorem{theorem}{定理}
\newtheorem{definition}{定义}
%\newtheorem{df}{定义}
\newtheorem{lemma}{引理}
\newtheorem{corollary}{推论}
\newtheorem{proposition}{性质}
\newtheorem{example}{例}
\newtheorem{remark}{注}
%%----------------------------------------------------------------------------------------------------
    \title{2016年秋《统计方法与应用》作业-3(随机变量)}
    \author{ 姓名:徐魁\,\,\,\, 学号~~{2016311209}}
    \date{\today}
\maketitle

%%---------------------------------------------------------------------------------------------------
\section{Reading. }
%===================================================================================================
\begin{enumerate}
  \item[(a)] Lecture notes 3.
 \item[(b)] Chpaters 4 of the book ”Statistical Inference”.\\
 
\end{enumerate}

%%---------------------------------------------------------------------------------------------------
\section{The joint distribution of X and Y is}
$$f(x,y)=\begin{cases}
	cx^2y, & \text{$x^2 \le y \le 1$}\\
	0, & \text{other}
\end{cases}$$
%===================================================================================================
\begin{enumerate}
  \item[(a)]  Calculate the constant $c$.
  解:$$\iint_{x^2 \le y \le 1} cx^2y\mathrm{d}x\mathrm{d}y
  =\int_{-1}^{1}\int_{x^2}^{1}cx^2y\mathrm{d}y\mathrm{d}x
  =\int_{-1}^{1}cx^2\frac{1}{2}(1-x^4)\mathrm{d}x
  =\int_{-1}^{1}\frac{c}{2}(x^2-x^6)\mathrm{d}x
  $$
  $$
  =\frac{c}{2}(\frac{1}{3}x^3 \vert_{-1}^{1}-\frac{1}{7}x^7 \vert_{-1}^{1})
  =\frac{4c}{21}=1
  $$
  因此,$c=\frac{21}{4}$
  
  %%-----------------------------b----- 
  \item[(b)]  解:先求$f_{X}(x)$\\
  $$f_{X}(x)=\int_{x^2}^{1}cx^2y\mathrm{d}y
  =cx^2 \frac{1}{2}y^2\vert_{x^2}^{y=1}
  =\frac{c}{2}x^2(1-x^4)=\frac{21}{8}x^2(1-x^6), 
  $$
  $x \in (-1,1]$\\
  因此,
  $$f_{X}(x)=\begin{cases}
	\frac{21}{8}x^2(1-x^6), & \text{$-1 \le x \le 1$}\\
	0, & \text{other}
   \end{cases}$$
   
   再求,$f_{Y}(y)$
   $$f_{Y}(y)=\int_{-\sqrt{y}}^{\sqrt{y}} cx^2y\mathrm{d}x
  =y  \frac{c}{3}x^3\vert_{-\sqrt{y}}^{\sqrt{y}}
  =\frac{2c}{3}y^{\frac{5}{2}}=\frac{7}{2}y^{\frac{5}{2}}, 
  $$
  $y \in [0,1]$\\
  因此,
  $$f_{Y}(y)=\begin{cases}
	\frac{7}{2}y^{\frac{5}{2}},  & \text{$0 \le y \le 1$}\\
	0, & \text{other}
   \end{cases}$$

  %%-----------------------------c-----
  \item[(c)]  解:
	$$F_{X|Y}(x,y)=\frac{f(x,y)}{f_{Y}^{x}}=\frac{cx^2y}{7/2y^{5/2}}=\frac{3}{2}x^2y^{-\frac{3}{2}}, 
	\begin{cases}
		\text{$0 \le y \le 1$}\\
		\text{$-\sqrt{y} \le x \le \sqrt{y}$}
   	\end{cases}
	$$
	因此,
	$$f_{X|Y}{(x,y)}=
	\begin{cases}
		\frac{3}{2}x^2y^{-\frac{3}{2}}, &	\text{$-\sqrt{y} \le x \le \sqrt{y}, 0 \le y \le 1$},\\
		0, & \text{other}
	\end{cases}
	$$
	当$Y=\frac{1}{2}$,$$f_{X|Y}{(x,y)}=
	\begin{cases}
		3\sqrt{2}x^2, &	\text{$-\frac{\sqrt{2}}{2} \le x \le \frac{\sqrt{2}}{2} $},\\
		0, & \text{other}
	\end{cases}
	$$
 
   %%-----------------------------d-----
  \item[(d)]  解:
	$$F_{X|Y}(\frac{1}{2},y)
	=\frac{c(1/2)^2y}{f_{X}^{1/2}}
	=\frac{\frac{21}{4}\frac{1}{4}y}{\frac{21}{8}(1/2)^2(1-(1/2)^4)}
	=\frac{32}{15}y, \frac{1}{4} \le y \le 1 
	$$
	因此,$$P(Y \ge \frac{1}{8}| X=\frac{1}{2})
	=\int_{\frac{1}{8}}^{1} \frac{32}{15}y \mathrm{d}y
	=\int_{\frac{1}{4}}^{1} \frac{32}{15}y \mathrm{d}y
	=1
	$$


\end{enumerate}
  
%%---------------------------------------------------------------------------------------------------
\section{Show that if $(X, Y ) \thicksim$ bivariate normal($\mu X , \mu Y , \sigma X^2 , \sigma Y^2 , \rho$), then the
following are true. }
%===================================================================================================


%%%%%%%%%%%%%%%%%%%%%%%%%%%%%%%%%%%%%%%%%%%%%%%%%%%%%%%%%%%%%%%%%%%%%%%%%%%%%%%%%%%%%%%%%%%%%%
  \end{CJK*}
\end{document}
