% !TEX encoding = UTF-8 Unicode
% !TEX TS-program = XeLaTeX
\documentclass [12pt]{article}
\usepackage[Lenny]{fncychap}
\usepackage{mathrsfs}
\usepackage{amsmath}
\usepackage{amssymb}
\usepackage{graphics}
\textwidth=6.5in \textheight=9in \topmargin=1pt \oddsidemargin=0pt
\evensidemargin=0pt
\renewcommand{\baselinestretch}{1.5}
\usepackage{CJK}
\usepackage{fontspec,xltxtra,xunicode}  
\defaultfontfeatures{Mapping=tex-text}  
%\setromanfont{SimSun} %设置中文字体  
\setromanfont{Kai} %设置中文字体  
\XeTeXlinebreaklocale “zh”  
\XeTeXlinebreakskip = 0pt plus 1pt minus 0.1pt %文章内中文自动换行  
\usepackage{multirow}


%-----------------------------------------------




%---------------------------------------------------
\def\hilite<#1>{\temporal<#1>{\color{blue!35}}{\color{magenta}}{\color{blue!75}}}
%% 自定义命令, 源自 beamer_guide. item 逐步显示时, 使已经出现的item、正在显示的item、将要出现的item 呈现不同颜色.
%%%%%%%%%%%%%%%%%%%%%%%%%%%%%%%%%%%%%%%%%%%%%%%%%%%%%%%%%%%%%%%%%%%%%%%%%%%%%%%%%%%%%%%%%%%%%%%%%%%%%%%%
\def\la{\lambda}
\def\bfta{{\bm{\theta}}}
\def\Ta{{\Theta}}
\def\ta{{\theta}}
\def\hta{{\hat{\theta}}}
\def\y1n{{Y_1,\cdots, Y_n}}
\def\x1n{{X_1,\cdots, X_n}}
\def\f12{{\frac{1}{2}}}
\def\barx{{\bar{X}}}
\def\f12{{\frac{1}{2}}}
\def\Sig{{\Sigma}}
\def\sig{{\sigma}}
\def\R{{\mathcal{R}}}
\def\p{{\mathcal{P}}}
\def\f{{\mathcal{F}}}
\def\e{{\epsilon}}
\def\barx{{\bar{X}}}
\def\bx{{\bm{x}}}
\def\bX{{\bm{\mathcal{X}}}}

\def\v#1{\stackrel{#1}{\longrightarrow}}
%---------------------------------------------
\newcommand{\cov}{\mbox{cov}}
\newcommand{\Var}{\mbox{Var}}
\newcommand{\E}{\mbox{E}}
%----------------------------------------------------
\newcommand{\limn}{{\lim\limits_{n\to\infty}}}
\newcommand{\liminfn}{{\liminf\limits_{n\to\infty}}}
\newcommand{\limsupn}{{\limsup\limits_{n\to\infty}}}

\newcommand{\sumx}[1]{\sum\limits_{i=1}^n {#1}_i}
\newcommand{\sumn}[1]{\sum\limits_{#1=1}^n}
%----------------------------------------------------
\newcommand{\timx}[1]{\prod\limits_{i=1}^n {#1}_i}
\newcommand{\timn}[1]{\prod\limits_{#1=1}^n}
%-------------------------------------------------------
\def\dps{\displaystyle}
\def\Om{{\Omega}}
\def\om{{\omega}}
\def\pp{{\mathcal{P}}}
\def\f{{\mathcal{F}}}
\def\R{{\mathcal{R}}}
\def\B{{\mathcal{B}}}
\def\A{{\mathcal{A}}}
\def\X{{\mathcal{X}}}
\def\C{{\mathcal{C}}}
\def\D{{\mathcal{D}}}
\def\O{{\mathcal{O}}}
\def\J{{\mathcal{J}}}
\def\Mu{{\mathcal{U}}}
%------------------------------------------
\def\b{{\beta}}
\def\a{{\alpha}}
\def\e{{\epsilon}}
\def\la{\lambda}
\def\La{\Lambda}
\def\ta{{\theta}}
\def\Ta{{\Theta}}
%-------------------------------------

\def\Sig{\Sigma}
\def\sig{\sigma}

\def\dfrac#1#2{{\displaystyle{#1\over#2}}}
\def\indA#1#2{{A^*_{#1 #2}(t)=a_m}}
\def\sumL {{\sum\limits_{m=1}^L}}
\def\dbN#1{{d \bar{N}(#1)}}


\def\Y#1#2{{Y_{#1#2}(t)}}
\def\Iam#1#2{{I(A^*_{#1#2}=a_m)}}


%------------------------------------------
\newcommand{\xn}[1]{ #1_1, \cdots, #1_n}
\newcommand{\capn}[1]{\cap\limits_{#1=1}^n}
\newcommand{\enorm}[2]{(E({#1}^#2))^{1/#2}}

%\newcommand{\sumx}[1]{\sum\limits_{i=1}^n {#1}_i}
\newcommand{\seqn}[1]{#1_1,#1_2,\cdots}

%%%%%%%%%%%%%%%%%%%%%%%%%%%%
\newtheorem{exam}{example}[section]
\newtheorem{df}{definition}[section]
\newtheorem{thm}{Theorem}[section]
\newtheorem{lem}{Lemma}[section]
\newtheorem{cor}{Corrolary}[section]
\newtheorem{rem}{Remark}[section]
%------------------------------------------------------------------
\begin{document}
 \begin{CJK*}{GBK}{kai}
 %%----------------------- Theorems ---------------------------------------------------------------------
\newtheorem{theorem}{定理}
\newtheorem{definition}{定义}
%\newtheorem{df}{定义}
\newtheorem{lemma}{引理}
\newtheorem{corollary}{推论}
\newtheorem{proposition}{性质}
\newtheorem{example}{例}
\newtheorem{remark}{注}
%%----------------------------------------------------------------------------------------------------
    \title{2016年秋《统计方法与应用》作业-2(随机变量)}
    \author{ 姓名:徐魁\,\,\,\, 学号~~{2016311209}}
    \date{\today}
\maketitle

%%---------------------------------------------------------------------------------------------------
\section{Reading. }
%===================================================================================================
\begin{enumerate}
  \item[(a)] Lecture notes 2.
 \item[(b)] Chpaters 2 and 3 of the book ”Statistical Inference”.\\
 
\end{enumerate}

%%---------------------------------------------------------------------------------------------------
\section{In each of the following show that the given function is a cdf and find $F_{X}^{-1}{(y)}$}
根据课本定理1.5.3,验证下列函数是否满足累积分布函数的三个性质即可。
\begin{enumerate}
  \item[(a)]  对于
$$F_{X}(x)=
\begin{cases}
0& \text{if x < 0}\\
1- e^{-x} & \text{if x $\ge$ 0}
\end{cases}$$
有:
\begin{itemize}
\item[-] $\lim\limits_{x \to -\infty }F(x) = 0$,因为$x<0$时,$F_{X}(x)=0$;又因为$\lim\limits_{x \to +\infty }e^{-x} = 0$, 所以$\lim\limits_{x \to +\infty }F(x) = 1$;
\item[-] 因为$e^x$是单调增函数,因此$1 - e^{-x}$还是单调增函数;或者可以证明导数大于0;直接求导有,$\frac{d}{dx}F_{X}(x)=(1-e^{-x})^{'}=-(e^{-x})^{'}(-x)^{'}=e^{-x} >0$
\item[-] 由于$F_{X}(x)$是连续函数,因此$F_{X}(x)$一定是右连续函数。
\end{itemize}
即得证。\\

求$F_{X}^{-1}(y)$:\\
有$y=1-e^{-x} $
 $\Leftrightarrow$\\
 $e^{-x}=1-y $
 $\Leftrightarrow$\\
 $\ln{e^{-x}}=\ln{1-y} $
$\Leftrightarrow$\\
 $x=- \ln{1-y} $\\
 即,$F_{X}^{-1}(y)=- \ln{(1-y)}$\\
 
 
  \item[(b)]  对于
$$F_{X}(x)=
\begin{cases}
e^x / 2 & \text{if x < 0}\\
1 / 2 & \text{if 0 $\le$ x < 1}\\
1- (e^{1-x} / 2) & \text{if 1 $\le$ x}
\end{cases}$$
有:
\begin{itemize}
\item[-] $\lim\limits_{x \to -\infty }F(x) = 0$,因为$x<0$时,$\lim\limits_{x \to -\infty }e^x = 0$;又因为$\lim\limits_{x \to +\infty }e^{1-x} = 0$, 所以$\lim\limits_{x \to +\infty }F(x) = 1$;
\item[-] 因为$e^x$是单调增函数,因此$1 - e^{-x}$还是单调增函数, 所以$1- (e^{1-x} / 2)$还是单调增函数,而$F_{X}(x)$在区间[0,1)是常数,常数是非单调递减,  因此$F_{X}(x)$在整个定义域上是单调增函数;
\item[-] 由于$F_{X}(x)$是连续函数,因为$\lim\limits_{x \to 0 }F(x) = 1/2$且$\lim\limits_{x \to 1 }F(x) = 1/2$,所以$F_{X}(x)$一定是右连续函数。
\end{itemize}
即得证。

求$F_{X}^{-1}(y)$:\\
有当$x \in (- \infty , 0)$时, 有$y \in [0,1/2]$,$y=e^x / 2 $(疑问2)
 $\Leftrightarrow$\\
 $e^x=2y $
 $\Leftrightarrow$\\
 $\ln{e^x}=\ln{(2y)} $
$\Leftrightarrow$\\
 $x= \ln{(2y)} $\\
 即,当$y \in [0,1/2]$时,$F_{X}^{-1}(y)= \ln{2y}$\\
 当$x \in [1 , + \infty)$时, 有$y \in [1/2,1)$,$y= 1- (e^{1-x} / 2) $
$\Leftrightarrow$\\
 $e^{1-x} =2(1-y) $
 $\Leftrightarrow$\\
 $\ln{e^{1-x}}=\ln{(2(1-y))} $
$\Leftrightarrow$\\
 $1-x= \ln{(2(1-y))} $\\
 $\Leftrightarrow$\\
 $x=1 - \ln{(2(1-y))} $\\
 即,当$y \in [1/2,1)$时,$F_{X}^{-1}(y)=1 - \ln{(2(1-y))} $\\
故,$$F_{X}^{-1}(y)=
\begin{cases}
ln{(2y)} & \text{if y $\in$ [0,1/2]}\\
1 - ln{(2(1-y))} & \text{if y $\in$ [1/2,1)}
\end{cases}$$
%%-----------------------------c-----
  \item[(c)]  对于
$$F_{X}(x)=
\begin{cases}
e^x / 4 & \text{if x < 0}\\
1- (e^{-x} / 4) & \text{if x $\ge$ 0}
\end{cases}$$
有:
\begin{itemize}
\item[-] $\lim\limits_{x \to -\infty }F(x) = 0$,因为$x<0$时,$\lim\limits_{x \to -\infty }e^x = 0$;又因为$\lim\limits_{x \to +\infty }e^{-x} = 0$, 所以$\lim\limits_{x \to +\infty }F(x) = 1$;
\item[-] 因为$e^x$是单调增函数,因此$1 - e^{-x}$还是单调增函数, 所以$1- (e^{-x} / 4)$还是单调增函数,而$F_{X}(x)$在区间[0,1)是常数,常数是非单调递减,  因此$F_{X}(x)$在整个定义域上是单调增函数;(略微有点疑问)
\item[-] 因为$\lim\limits_{x \to 0^{+} }F(x) = 3/4$且$F(0) = 3/4$,所以$F_{X}(x)$是右连续函数。
\end{itemize}
即得证。

求$F_{X}^{-1}(y)$:\\
有当$x \in (- \infty , 0)$时, 有$y \in [0,1/4)$,$y=e^x / 4$
 $\Leftrightarrow$\\
 $e^x=4y $
 $\Leftrightarrow$\\
 $\ln{e^x}=\ln{(4y)} $
$\Leftrightarrow$\\
 $x= \ln{(4y)} $\\
 即,当$y \in [0,1/4)$时,$F_{X}^{-1}(y)= \ln{(4y)}$\\
 当$x \in [0 , + \infty)$时, 有$y \in [3/4,1)$,$y= 1- (e^{-x} / 4) $
  $\Leftrightarrow$\\
 $e^{-x}=4(1-y) $
 $\Leftrightarrow$\\
 $\ln{e^{-x}}=\ln{4(1-y)} $
$\Leftrightarrow$\\
 $x=- \ln{(4(1-y))} $\\ 即,当$y \in [1/4,1)$时,$F_{X}^{-1}(y)=1 - \ln{(4(1-y))} $\\
故,$$F_{X}^{-1}(y)=
\begin{cases}
ln{(4y)} & \text{if y $\in$ [0,1/4)}\\
1 - ln{(4(1-y))} & \text{if y $\in$ [3/4,1)}
\end{cases}$$

\end{enumerate}
  
%%---------------------------------------------------------------------------------------------------
\section{Let $X$ have the pdf,  $$f(x)=\frac{4}{\beta^{3}\sqrt{\pi}}x^{2}e^{-x^2/\beta^{2}},x \in [0,\infty), \beta >0$$ }
%===================================================================================================
   \begin{enumerate}
      \item[(a)] Verify $f(x)$ is a valid pdf.\\
      证明:两个性质不难证明性质a, 即$f_{X}(x) \ge 0$。\\
      不难推导,$$f(x)=\int_{0}^{\infty}f(x) dx = \int_{0}^{\infty}\frac{4}{\beta^{3}\sqrt{\pi}}x^{2}e^{-x^2/\beta^{2}} dx \approx 1$$ 即可证$f(x)$ 是概率密度函数。
      \item[(b)] Find  $\mathbb{E}(X)$ and $Var X$.\\
      解: 
      首先,因为,
$$ \int_{0}^\infty xe^{-x} dx=\Gamma{(2)} =1;
$$      先求期望,$$\mathbb{E}X=\int_{0}^{\infty}{\frac{4}{\beta^{3}\sqrt{\pi}}x^{3}e^{-x^2/\beta^{2}}}dx $$
      令$t=x/\beta$, 有$$\mathbb{E}X=\frac{4\beta}{\sqrt{\pi}}  \int_{0}^{\infty}{t^{3}e^{-t^{2}}}dt $$\\
      再令$m=t^2$,有$$\mathbb{E}X=\frac{4\beta}{\sqrt{\pi}} \frac{1}{2} \int_{0}^{\infty}{ m e^{-m}}dm $$
      进而,$$\mathbb{E}X=\frac{2\beta}{\sqrt{\pi}} \int_{0}^{\infty}{ m e^{-m}}dm = \frac{2\beta}{\sqrt{\pi}}$$
      	再求平方的期望,因为$$\int_{-\infty}^\infty x^2e^{-x^2} dx=\int_{0}^\infty xe^{-x^2} 2xdx=\int_{0}^\infty u^{\frac{1}{2}} e^{-u}du=\Gamma\left(\frac{3}{2}\right) =\frac{\sqrt{\pi}}{2}$$, 且这个函数是关于0对称,因此$$ \int_{-\infty}^\infty x^2e^{-x^2} dx=\frac{\sqrt{\pi}}{4}$$
	
	
	$$\mathbb{E}X=\int_{0}^{\infty}{\frac{4}{\beta^{3}\sqrt{\pi}}x^{4}e^{-x^2/\beta^{2}}}dx $$
      令$t=x/\beta$, 有$$\mathbb{E}X=\frac{4\beta^2}{\sqrt{\pi}}  \int_{0}^{\infty}{t^{4}e^{-t^{2}}}dt $$\\
     $$\mathbb{E}X=\frac{4\beta^2}{\sqrt{\pi}} (-\frac{1}{2}) \int_{0}^{\infty}{ t^3 e^{-t^2}}d(-t^2) $$
      进而,$$\mathbb{E}X=-\frac{2\beta^2}{\sqrt{\pi}} \int_{0}^{\infty}{  t^3 d(e^{-t^2})} = -\frac{2\beta^2}{\sqrt{\pi}} \left(  t^3 e^{-t^2} \vert_{0}^{+\infty }  - \int_{0}^{\infty}{ e^{-t^2}}d(t^3) \right) $$
      $$= -\frac{2\beta^2}{\sqrt{\pi}} \left(  t^3 e^{-t^2} \vert_{0}^{+\infty }  - \int_{0}^{\infty}{3 t^2 e^{-t^2}}dt \right) $$
      $$= -\frac{2\beta^2}{\sqrt{\pi}} \left(-3 \frac{\sqrt{\pi}}{4} \right)  = \frac{3\beta^2}{2} $$
	因此方差$VarX=\mathbb{E}(x^2)-(\mathbb{E}(x))^2 =\frac{3\beta^2}{2} - (\frac{2\beta}{\sqrt{\pi}})^2$
	
\end{enumerate}	
%%---------------------------------------------------------------------------------------------------
\section{证明}
%===================================================================================================
 \begin{enumerate}
  \item[(a)] 设X是连续且非负的随机变量,证明$EX=\int_{0}^{\infty}{[1-F_{X}(x)]dx}$\\
  证明:由于$F_{X}(x)=P(X \le x)$, 且,$1-F_{X}(x)=P(X > x)$\\
  %每一步的详细的依据可以放在上面
  
  那么,有$$\int_{0}^{\infty}{(1-F_{X}(x))dx}=\int_{0}^{\infty}{P(X > x)dx}$$\\
  而根据定义$$EX= \int_{0}^{\infty}{xf_{X}(x)dx}$$
  $$ = \int_{0}^{\infty}{\int_{x}^{\infty}{f_{X}(x)dydx}}$$
  $$ = \int_{0}^{\infty}{\int_{0}^{y}{dxf_{X}(y)dy}}$$ 
  $$ = \int_{0}^{\infty}{yf_{X}(y)dy}$$ 
  $$ = \int_{0}^{\infty}{xf_{X}(x)dx}$$ 
  即$$ = EX$$ 
  故得证。
  
  \item[(b)] 设X是取值为非负整数的离散随机变量,证明:$EX=\sum_{k=0}^{\infty}{(1-F_{X}(k))}$\\
  %离散型的如何求?
  证明:
\end{enumerate}


 %%---------------------------------------------------------------------------------------------------
\section{ 设$f(x)$为一概率密度函数,如果存在数$a$使得:对于任意$\varepsilon > 0$都有$f(a+\varepsilon)=f(a-\varepsilon)$, 则称$f(x)$关于a对称。}
%===================================================================================================
  \begin{enumerate}
  \item[(a)] 三个对称的概率密度函数:
 	 \begin{itemize}
		\item[-] 正态分布:$$f(x)=\frac{1}{\sqrt{2\pi}}e^{-x^2/2}$$
		\item[-] 柯西分布:$$f(x)=\frac{1}{\pi(1+x^2)}$$
		\item[-] 罗吉斯蒂克概率函数: $$f(x)=\frac{1}{1+e^{-x}}$$
	\end{itemize}

  	
  \item[(b)] 因为概率分布的中位数满足$P(X \le m) \ge \frac{1}{2}$ 且$P(X \ge m) \ge \frac{1}{2}$ ,$$\int_{-\infty}^af(x)dx=\int_a^{\infty}f(2a-x)dx=1/2$$.\\
  	令$\varepsilon=x-a$那么有,$$\int_{a}^{\infty}f(x)dx=\int_{0}^{\infty}f(a+\varepsilon)d\varepsilon=\int_{0}^{\infty}f(a-\varepsilon)d\varepsilon$$\\
	令$x=a-\varepsilon$那么有上式等于$$=\int_{-\infty}^{a}f(x)dx$$\\
	即$$\int_{a}^{\infty}f(x)dx=\int_{-\infty}^{a}f(x)dx$$
	而对于一个概率密度函数有$$\int_{a}^{\infty}f(x)dx + \int_{-\infty}^{a}f(x)dx=1$$,
	因此有$$\int_{a}^{\infty}f(x)dx=\int_{-\infty}^{a}f(x)dx=\frac{1}{2}$$,\\
	那么根据中位数的性质,可得该函数的中位数就是$a$。
  \item[(c)] 根据期望的定义可得,$$EX=\int_{-\infty}^{\infty}xf(x)dx$$\\
  	且有,$$EX- a=E(X-a)$$\\
	因此$$EX- a=E(X-a)=\int_{-\infty}^{\infty}(x-a)f(x)dx$$\\
	此时令$\varepsilon=x-a$,有上式等于$$=\int_{0}^{\infty}-\varepsilon f(a-\varepsilon)d\varepsilon+\int_{0}^{\infty}\varepsilon f(a+\varepsilon)d\varepsilon$$\\
  	又有对称函数的性质,$f(a+\varepsilon)=f(a-\varepsilon)$可得,上式为0\\
	即$$EX- a=0$$, 因此有$$EX= a$$
  \item[(d)] 对于$f(x)=e^{-x}$有,$f(a+\varepsilon)=e^{-a-\varepsilon}$,$f(a-\varepsilon)=e^{-a+\varepsilon}$,\\
  	可得$\frac{f(a+\varepsilon)}{f(a-\varepsilon)}=\frac{e^-\varepsilon}{e^\varepsilon}=\frac{1}{e^{2\varepsilon}}$\\
	因为$\varepsilon \ge 0$, 因此$\frac{1}{e^{2\varepsilon}} \ne 1 $, 因此$f(x)=e^{-x}$不是对称的概率密度函数。
 \item[(e)] 对于$f(x)=e^{-x}$,可求得中值为$log(2)$, 而期望$EX=\int_{-\infty}^\infty{ xf(x)dx}=1$\\
 	即中位数小于期望。
\end{enumerate}






%%---------------------------------------------------------------------------------------------------
\section{求下列分布的矩母函数}
%===================================================================================================
\begin{enumerate}
  \item[(a)]  $f(x)=\frac{1}{c}, 0<x<c$; \\
	  解:根据矩母函数的定义,有:\\
	  $$Ee^{tX}=\int_{0}^{c}e^{tx}f(x)dx$$\\
	  $$=\int_{0}^{c}e^{tx}\frac{1}{c}dx=\frac{1}{ct}e^{tx} \vert_{0}^{c} =\frac{1}{ct}(e^{tc}-1)$$
  \item[(b)]  $f(x)=\frac{2x}{c^2},  0<x<c$;\\
  	 解:根据矩母函数的定义,有:\\
	  $$Ee^{tX}=\int_{0}^{c}e^{tx}f(x)dx$$\\
	  $$=\int_{0}^{c}e^{tx}\frac{2x}{c^2}dx=\frac{2x}{c^2}e^{tx} \vert_{0}^{c} =\frac{2}{c^2 t^2}(cte^{tc}-e^{tc}+1)$$
  \item[(c)]  $f(x)=\frac{1}{2\beta}e^{-|x-\alpha|/\beta},  -\infty<x<\infty, -\infty<\alpha<\infty,\beta>0$;\\
  	 解:根据矩母函数的定义,有:\\
	  $$Ee^{tX}=\int_{-\infty}^{\infty}e^{tx}f(x)dx$$\\
	  $$=\int_{-\infty}^{\infty}e^{tx}\frac{1}{2\beta}e^{-|x-\alpha|/\beta}dx$$\\
	  $$=\int_{-\infty}^{\alpha}e^{tX}\frac{1}{2\beta}e^{(x-\alpha)/\beta}dx+\int_{\alpha}^{\alpha}e^{tX}\frac{1}{2\beta}e^{-(x-\alpha)/\beta}dx$$
	  $$=\frac{e^{-\alpha/\beta}}{2\beta} \frac{1}{1/\beta+t} e^{-x(1/\beta+t)}\vert_{-\infty}^{\alpha}- \frac{e^{\alpha/\beta}}{2\beta} \frac{1}{1/\beta-t} e^{-x(1/\beta-t)}\vert_{\alpha}^{\infty}$$
	  $$=\frac{4}{4-\beta^2 t^2}e^{\alpha t}$$
 
  
 \end{enumerate}  

%%---------------------------------------------------------------------------------------------------
\section{求出下列$Y$的概率密度函数}
%===================================================================================================
\begin{enumerate}
  \item[(a)]  $Y=X^2$ and $f_{X}(x)=1, 0<x<1$\\
  	解:令$Y=g(x)$,则$g^{-1}(y)=y^{1/2}$且$\frac{d}{dy}g^{-1}(y)=\frac{1}{2\sqrt{y}}$\\
  	对于,$0<x<1$,有$Y=g(x)$是单调增函数,因此由课本定理2.1.5可得,概率密度函数$f_{Y}(y)=f_{X}(g^{-1}(y))\frac{d}{dy}g^{-1}(y)$, $0<y<1$\\
	即$$f_{Y}(y)=1 * \frac{1}{2\sqrt{y}}= \frac{1}{2\sqrt{y}}$$,且 $0<y<1$。
	
  \item[(b)]  $Y=-log(X)$ and $X$ has pdf, $f_{X}(x)=\frac{(m+n+1)!}{n!m!}x^n(1-x)^m, 0<x<1,m,n$ 为正整数。\\
  	解:令$Y=g(x)$,则$g^{-1}(y)=e^{-y}$且$\frac{d}{dy}g^{-1}(y)=-e^{-y}$\\
  	对于,$0<x<1$,有$Y=g(x)$是单调减函数,因此由课本定理2.1.5可得,概率密度函数$f_{Y}(y)=-f_{X}(g^{-1}(y))\frac{d}{dy}g^{-1}(y)$, $0<y<1$\\
	即$$f_{Y}(y)=- \frac{(m+n+1)!}{n!m!} (e^{-y})^n(1-e^{-y})^m  -e^{-y}= \frac{(m+n+1)!}{n!m!}e^{-y(n+1)}(1-e^{-y})^m$$, $0<y<\infty$。

  \item[(c)]  $Y=e^X$ and $X$ has pdf, $f_{X}(x)=\frac{1}{\sigma^2}xe^{-(x/\sigma)^2/2}$, $0<x<\infty$, $ \sigma^2$为正数。\\
  	解:令$Y=g(x)$,则$g^{-1}(y)=\log y$且$\frac{d}{dy}g^{-1}(y)=1/y$\\
  	对于,$0<x<1$,有$Y=g(x)$是单调增函数,因此由课本定理2.1.5可得,概率密度函数$f_{Y}(y)=f_{X}(g^{-1}(y))\frac{d}{dy}g^{-1}(y)$, $0<y<1$\\
	即$$f_{Y}(y)=\frac{1}{\sigma^2}(\log{y})e^{-(\log{y}/\sigma)^2/2}  * (1/y)= \frac{\log{y}}{y\sigma^2}e^{-(\log{y}/\sigma)^2/2}  $$, $0<y<\infty$。

\end{enumerate}  

%%---------------------------------------------------------------------------------------------------

\section{A random variable $X$ is said to have a Gamma distribution if its pdf is: \\
     $$f(x|shape=\alpha,scale=\theta)=\frac{1}{\Gamma(\alpha)\theta^\alpha}x^{\alpha-1}e^{-x/\theta},x \in [0,\infty), \alpha >0,\theta>0$$}
%===================================================================================================
\begin{enumerate}      
	\item[(a)] Verify $f(x|\alpha,\theta)$ is a valid pdf.\\
		证明:两个性质不难证明性质a, 即$f_{X}(x) \ge 0$。\\
      		不难推导,$$f_{X}(x|\alpha,\theta)=\int_{-\infty}^{\infty}f_{X}(x|\alpha,\theta) dx $$
		$$= \int_{-\infty}^{\infty}\frac{1}{\Gamma(\alpha)\theta^\alpha}x^{\alpha-1}e^{-x/\theta} dx \approx 1$$ 
		即可证$f(x|\alpha,\theta)$ 是概率密度函数。
		
      	\item[(b)] Find the mode of a Gamma random variable (for $\alpha$ > 1); \\
		解:当 $\alpha$ > 1时,$f(x)$先递增,后递减,mode为$(\alpha -1)\theta$
	\item[(c)] Find the moment generating function M(t) of a Gamma random variable;\\
		解:根据$\Gamma(\alpha)$函数的性质,其对应的矩母函数为:
		$$M_{X}(t)=\frac{1}{\Gamma(\alpha)\theta^\alpha} \int_{0}^{+\infty} e^{tx} x^{\alpha-1}e^{-x/\theta}dx$$
		$$=\frac{1}{\Gamma(\alpha)\theta^\alpha} \int_{0}^{+\infty} x^{\alpha-1}e^{-(1/\theta-t)x}dx$$
		根据伽玛函数的性质,对于任意大于0的常数$\alpha,\beta$:
		$$f(x)=\frac{1}{\Gamma(a)b^a}x^{a-1}e^{-x/b}$$,都是某随机变量的概率密度函数,于是
		$$\int_{0}^{+\infty}\frac{1}{\Gamma(a)b^a}x^{a-1}e^{-x/b}dx=1$$
		也就是$$\int_{0}^{+\infty}x^{a-1}e^{-x/b}dx=\Gamma(a)b^a$$
		即得:当$t<1/\theta$有
		$$M_{X}(t)=\frac{1}{\Gamma(\alpha)\theta^\alpha} \Gamma(\alpha)\left( \frac{\theta}{1-\theta t}\right)^\alpha=\left( \frac{1}{1-\theta t}\right)^\alpha$$
		而当$t\ge 1/\theta$时,没有矩母函数,因为$M_{X}(x)积分为无穷$。
	\item[(d)] Find \textbf{mean} , \textbf{variance} of $X$, the  \textbf{skewness} and the  \textbf{kurtosis} of a Gamma random variable;
      		解: 先求期望,$$EX=\frac{d}{dx}M_{X}(t)\vert_{t=0}=\frac{\alpha\theta}{(1-\theta t)^{\alpha+1}}\vert_{t=0}=\alpha\theta$$
      		再求平方的期望,$$EX^2=\frac{d^{(2)}}{dx}M_{X}^{(2)}(t)\vert_{t=0}=\frac{d}{dx}\frac{\alpha\theta}{(1-\theta t)^{\alpha+1}}\vert_{t=0}=\frac{(\alpha+1)\alpha\theta^2}{(1-\theta t)^{\alpha+2}}\vert_{t=0}=(\alpha+1)\alpha\theta^2$$,
		因此方差$VarX=EX^2-(EX)^2 =\alpha\theta^2$\\
		The skewness of a random variable X is its third central moment ,因此$$EX^3=\frac{d^{(3)}}{dx}M_{X}^{(2)}(t)\vert_{t=0}=\frac{2}{\sqrt{\alpha}}$$\\
		The Kurtosis of a random variable X is its fourth central moment,因此$$EX^4=\frac{d^{(4)}}{dx}M_{X}^{(2)}(t)\vert_{t=0}=3+\frac{6}{\alpha}$$\\
		
      	\item[(e)] Let $Y=1/X$. What is the pdf of $Y$?($Y$ is said to have an inverse gamma distribution)
      		解:令$Y=1/X$, 则有,$\frac{dx}{dy}=\frac{1}{y^2}$, 且$Y$的概率密度函数为:\\
      		$$f_{Y}(y|\alpha,\theta)=f_{X}(y|\alpha,\theta) \frac{dx}{dy}$$
      		$$f_{Y}(y|\alpha,\theta)=\frac{1}{\Gamma(\alpha)\theta^\alpha}(\frac{1}{y})^{\alpha-1}e^{-1/(\theta y)} \mid  \frac{1}{y^2}\mid$$
      		$$f_{Y}(y|\alpha,\theta)=\frac{1}{\Gamma(\alpha)\theta^\alpha}y^{-\alpha-1}e^{-1/(\theta y)} $$
      		用$\beta$替换$\theta^{-1}$得:
       		$$f_{Y}(y|\alpha,\beta)=\frac{\beta^{\alpha}}{\Gamma(\alpha)}y^{-\alpha-1}e^{-\beta/y} $$

      		因此,$Y$的概率密度函数为逆Gamma分布。
      \end{enumerate}

 %%---------------------------------------------------------------------------------------------------
\section{A random variable X is said to have a Possion distribution if its pdf is: \\
     $$f(x|shape=\alpha,scale=\theta)=\frac{1}{\Gamma(\alpha)\theta^\alpha}x^{\alpha-1}e^{-x/\theta},x \in [0,\infty), \alpha >0,\theta>0$$}
%===================================================================================================
\begin{enumerate}      
	\item[(a)] Verify $f(X=k)$ is a valid pdf.\\
		证明:两个性质不难证明性质a, 即$f(X=k) \ge 0$。\\
      		不难推导,$$f(X=k)=\frac{\lambda^k}{k!}e^{-\lambda},k=0,1,2,... $$
		$$= \sum_{x=0}^{\infty}P(X=x|\lambda)=e^{-\lambda}\sum_{x=0}^{\infty}\frac{\lambda^x}{x!}=e^{-\lambda} e^\lambda=1$$ 
		即可证$f(X=k)$ 是概率密度函数。
	\item[(b)] Find the moment generating function M(t) of a Gamma random variable;\\
		解:矩母函数为:
		$$M_{X}(t)= \sum_{x=0}^{\infty}e^{tx}\frac{e^{-\lambda}\lambda^x}{x!}$$
		$$=e^{-\lambda}\sum_{x=1}^{\infty}\frac{(e^t\lambda)^x}{x!}=e^{-\lambda} e^{\lambda e^t}$$
		$$=e^{\lambda(e^t-1)}$$
		
	\item[(c)]	Find the mean, the variance, the skewness and the kurtosis of a Poisson random variable;
		 解: 先求期望,$$EX=\frac{d}{dx}M_{X}(t)\vert_{t=0}=e^{\lambda(e^t-1)}\lambda e^t \vert_{t=0}=\lambda$$
		 再求平方的期望,$$EX^2=\frac{d^{(2)}}{dx}M_{X}^{(2)}(t)\vert_{t=0}=\lambda^2+\lambda$$,
		因此方差$VarX=EX^2-(EX)^2 =\lambda^2+\lambda-\lambda^2=\lambda$\\
		The skewness of a random variable X is its third central moment ,因此$$EX^3=\frac{d^{(3)}}{dx}M_{X}^{(2)}(t)\vert_{t=0}=\lambda^{-1/2}$$\\
		The Kurtosis of a random variable X is its fourth central moment,因此$$EX^4=\frac{d^{(4)}}{dx}M_{X}^{(2)}(t)\vert_{t=0}=\lambda^{-1}$$\\
  \end{enumerate}

\section{Show that\\
     $$\int_{x}^{\infty}\frac{1}{\Gamma(\alpha)}z^{\alpha-1}e^{-z}dz=\sum_{y=0}^{\alpha-1}\frac{x^ye^{-x}}{y!},\alpha=1,2,3,...$$}
     (Hint:Use integration by parts.) Express this formula as a probabilistic relationship between Possion and gamma random variables.
%===================================================================================================
	解:对左式进行分部积分, $$\int_{x}^{\infty}\frac{1}{\Gamma(\alpha)}z^{\alpha-1}e^{-z}dz$$


%%%%%%%%%%%%%%%%%%%%%%%%%%%%%%%%%%%%%%%%%%%%%%%%%%%%%%%%%%%%%%%%%%%%%%%%%%%%%%%%%%%%%%%%%%%%%%
  \end{CJK*}
\end{document}
